% Generated by Sphinx.
\def\sphinxdocclass{report}
\documentclass[letterpaper,10pt,english]{sphinxmanual}
\usepackage[utf8]{inputenc}
\DeclareUnicodeCharacter{00A0}{\nobreakspace}
\usepackage{cmap}
\usepackage[T1]{fontenc}
\usepackage{babel}
\usepackage{times}
\usepackage[Bjarne]{fncychap}
\usepackage{longtable}
\usepackage{sphinx}
\usepackage{multirow}

\addto\captionsenglish{\renewcommand{\figurename}{Fig. }}
\addto\captionsenglish{\renewcommand{\tablename}{Table }}
\floatname{literal-block}{Listing }



\title{OraMon4PRTG Documentation}
\date{September 21, 2015}
\release{1.0}
\author{David Eugenio Perez Negron Rocha}
\newcommand{\sphinxlogo}{}
\renewcommand{\releasename}{Release}
\makeindex

\makeatletter
\def\PYG@reset{\let\PYG@it=\relax \let\PYG@bf=\relax%
    \let\PYG@ul=\relax \let\PYG@tc=\relax%
    \let\PYG@bc=\relax \let\PYG@ff=\relax}
\def\PYG@tok#1{\csname PYG@tok@#1\endcsname}
\def\PYG@toks#1+{\ifx\relax#1\empty\else%
    \PYG@tok{#1}\expandafter\PYG@toks\fi}
\def\PYG@do#1{\PYG@bc{\PYG@tc{\PYG@ul{%
    \PYG@it{\PYG@bf{\PYG@ff{#1}}}}}}}
\def\PYG#1#2{\PYG@reset\PYG@toks#1+\relax+\PYG@do{#2}}

\expandafter\def\csname PYG@tok@gd\endcsname{\def\PYG@tc##1{\textcolor[rgb]{0.63,0.00,0.00}{##1}}}
\expandafter\def\csname PYG@tok@gu\endcsname{\let\PYG@bf=\textbf\def\PYG@tc##1{\textcolor[rgb]{0.50,0.00,0.50}{##1}}}
\expandafter\def\csname PYG@tok@gt\endcsname{\def\PYG@tc##1{\textcolor[rgb]{0.00,0.27,0.87}{##1}}}
\expandafter\def\csname PYG@tok@gs\endcsname{\let\PYG@bf=\textbf}
\expandafter\def\csname PYG@tok@gr\endcsname{\def\PYG@tc##1{\textcolor[rgb]{1.00,0.00,0.00}{##1}}}
\expandafter\def\csname PYG@tok@cm\endcsname{\let\PYG@it=\textit\def\PYG@tc##1{\textcolor[rgb]{0.25,0.50,0.56}{##1}}}
\expandafter\def\csname PYG@tok@vg\endcsname{\def\PYG@tc##1{\textcolor[rgb]{0.73,0.38,0.84}{##1}}}
\expandafter\def\csname PYG@tok@m\endcsname{\def\PYG@tc##1{\textcolor[rgb]{0.13,0.50,0.31}{##1}}}
\expandafter\def\csname PYG@tok@mh\endcsname{\def\PYG@tc##1{\textcolor[rgb]{0.13,0.50,0.31}{##1}}}
\expandafter\def\csname PYG@tok@cs\endcsname{\def\PYG@tc##1{\textcolor[rgb]{0.25,0.50,0.56}{##1}}\def\PYG@bc##1{\setlength{\fboxsep}{0pt}\colorbox[rgb]{1.00,0.94,0.94}{\strut ##1}}}
\expandafter\def\csname PYG@tok@ge\endcsname{\let\PYG@it=\textit}
\expandafter\def\csname PYG@tok@vc\endcsname{\def\PYG@tc##1{\textcolor[rgb]{0.73,0.38,0.84}{##1}}}
\expandafter\def\csname PYG@tok@il\endcsname{\def\PYG@tc##1{\textcolor[rgb]{0.13,0.50,0.31}{##1}}}
\expandafter\def\csname PYG@tok@go\endcsname{\def\PYG@tc##1{\textcolor[rgb]{0.20,0.20,0.20}{##1}}}
\expandafter\def\csname PYG@tok@cp\endcsname{\def\PYG@tc##1{\textcolor[rgb]{0.00,0.44,0.13}{##1}}}
\expandafter\def\csname PYG@tok@gi\endcsname{\def\PYG@tc##1{\textcolor[rgb]{0.00,0.63,0.00}{##1}}}
\expandafter\def\csname PYG@tok@gh\endcsname{\let\PYG@bf=\textbf\def\PYG@tc##1{\textcolor[rgb]{0.00,0.00,0.50}{##1}}}
\expandafter\def\csname PYG@tok@ni\endcsname{\let\PYG@bf=\textbf\def\PYG@tc##1{\textcolor[rgb]{0.84,0.33,0.22}{##1}}}
\expandafter\def\csname PYG@tok@nl\endcsname{\let\PYG@bf=\textbf\def\PYG@tc##1{\textcolor[rgb]{0.00,0.13,0.44}{##1}}}
\expandafter\def\csname PYG@tok@nn\endcsname{\let\PYG@bf=\textbf\def\PYG@tc##1{\textcolor[rgb]{0.05,0.52,0.71}{##1}}}
\expandafter\def\csname PYG@tok@no\endcsname{\def\PYG@tc##1{\textcolor[rgb]{0.38,0.68,0.84}{##1}}}
\expandafter\def\csname PYG@tok@na\endcsname{\def\PYG@tc##1{\textcolor[rgb]{0.25,0.44,0.63}{##1}}}
\expandafter\def\csname PYG@tok@nb\endcsname{\def\PYG@tc##1{\textcolor[rgb]{0.00,0.44,0.13}{##1}}}
\expandafter\def\csname PYG@tok@nc\endcsname{\let\PYG@bf=\textbf\def\PYG@tc##1{\textcolor[rgb]{0.05,0.52,0.71}{##1}}}
\expandafter\def\csname PYG@tok@nd\endcsname{\let\PYG@bf=\textbf\def\PYG@tc##1{\textcolor[rgb]{0.33,0.33,0.33}{##1}}}
\expandafter\def\csname PYG@tok@ne\endcsname{\def\PYG@tc##1{\textcolor[rgb]{0.00,0.44,0.13}{##1}}}
\expandafter\def\csname PYG@tok@nf\endcsname{\def\PYG@tc##1{\textcolor[rgb]{0.02,0.16,0.49}{##1}}}
\expandafter\def\csname PYG@tok@si\endcsname{\let\PYG@it=\textit\def\PYG@tc##1{\textcolor[rgb]{0.44,0.63,0.82}{##1}}}
\expandafter\def\csname PYG@tok@s2\endcsname{\def\PYG@tc##1{\textcolor[rgb]{0.25,0.44,0.63}{##1}}}
\expandafter\def\csname PYG@tok@vi\endcsname{\def\PYG@tc##1{\textcolor[rgb]{0.73,0.38,0.84}{##1}}}
\expandafter\def\csname PYG@tok@nt\endcsname{\let\PYG@bf=\textbf\def\PYG@tc##1{\textcolor[rgb]{0.02,0.16,0.45}{##1}}}
\expandafter\def\csname PYG@tok@nv\endcsname{\def\PYG@tc##1{\textcolor[rgb]{0.73,0.38,0.84}{##1}}}
\expandafter\def\csname PYG@tok@s1\endcsname{\def\PYG@tc##1{\textcolor[rgb]{0.25,0.44,0.63}{##1}}}
\expandafter\def\csname PYG@tok@gp\endcsname{\let\PYG@bf=\textbf\def\PYG@tc##1{\textcolor[rgb]{0.78,0.36,0.04}{##1}}}
\expandafter\def\csname PYG@tok@sh\endcsname{\def\PYG@tc##1{\textcolor[rgb]{0.25,0.44,0.63}{##1}}}
\expandafter\def\csname PYG@tok@ow\endcsname{\let\PYG@bf=\textbf\def\PYG@tc##1{\textcolor[rgb]{0.00,0.44,0.13}{##1}}}
\expandafter\def\csname PYG@tok@sx\endcsname{\def\PYG@tc##1{\textcolor[rgb]{0.78,0.36,0.04}{##1}}}
\expandafter\def\csname PYG@tok@bp\endcsname{\def\PYG@tc##1{\textcolor[rgb]{0.00,0.44,0.13}{##1}}}
\expandafter\def\csname PYG@tok@c1\endcsname{\let\PYG@it=\textit\def\PYG@tc##1{\textcolor[rgb]{0.25,0.50,0.56}{##1}}}
\expandafter\def\csname PYG@tok@kc\endcsname{\let\PYG@bf=\textbf\def\PYG@tc##1{\textcolor[rgb]{0.00,0.44,0.13}{##1}}}
\expandafter\def\csname PYG@tok@c\endcsname{\let\PYG@it=\textit\def\PYG@tc##1{\textcolor[rgb]{0.25,0.50,0.56}{##1}}}
\expandafter\def\csname PYG@tok@mf\endcsname{\def\PYG@tc##1{\textcolor[rgb]{0.13,0.50,0.31}{##1}}}
\expandafter\def\csname PYG@tok@err\endcsname{\def\PYG@bc##1{\setlength{\fboxsep}{0pt}\fcolorbox[rgb]{1.00,0.00,0.00}{1,1,1}{\strut ##1}}}
\expandafter\def\csname PYG@tok@mb\endcsname{\def\PYG@tc##1{\textcolor[rgb]{0.13,0.50,0.31}{##1}}}
\expandafter\def\csname PYG@tok@ss\endcsname{\def\PYG@tc##1{\textcolor[rgb]{0.32,0.47,0.09}{##1}}}
\expandafter\def\csname PYG@tok@sr\endcsname{\def\PYG@tc##1{\textcolor[rgb]{0.14,0.33,0.53}{##1}}}
\expandafter\def\csname PYG@tok@mo\endcsname{\def\PYG@tc##1{\textcolor[rgb]{0.13,0.50,0.31}{##1}}}
\expandafter\def\csname PYG@tok@kd\endcsname{\let\PYG@bf=\textbf\def\PYG@tc##1{\textcolor[rgb]{0.00,0.44,0.13}{##1}}}
\expandafter\def\csname PYG@tok@mi\endcsname{\def\PYG@tc##1{\textcolor[rgb]{0.13,0.50,0.31}{##1}}}
\expandafter\def\csname PYG@tok@kn\endcsname{\let\PYG@bf=\textbf\def\PYG@tc##1{\textcolor[rgb]{0.00,0.44,0.13}{##1}}}
\expandafter\def\csname PYG@tok@o\endcsname{\def\PYG@tc##1{\textcolor[rgb]{0.40,0.40,0.40}{##1}}}
\expandafter\def\csname PYG@tok@kr\endcsname{\let\PYG@bf=\textbf\def\PYG@tc##1{\textcolor[rgb]{0.00,0.44,0.13}{##1}}}
\expandafter\def\csname PYG@tok@s\endcsname{\def\PYG@tc##1{\textcolor[rgb]{0.25,0.44,0.63}{##1}}}
\expandafter\def\csname PYG@tok@kp\endcsname{\def\PYG@tc##1{\textcolor[rgb]{0.00,0.44,0.13}{##1}}}
\expandafter\def\csname PYG@tok@w\endcsname{\def\PYG@tc##1{\textcolor[rgb]{0.73,0.73,0.73}{##1}}}
\expandafter\def\csname PYG@tok@kt\endcsname{\def\PYG@tc##1{\textcolor[rgb]{0.56,0.13,0.00}{##1}}}
\expandafter\def\csname PYG@tok@sc\endcsname{\def\PYG@tc##1{\textcolor[rgb]{0.25,0.44,0.63}{##1}}}
\expandafter\def\csname PYG@tok@sb\endcsname{\def\PYG@tc##1{\textcolor[rgb]{0.25,0.44,0.63}{##1}}}
\expandafter\def\csname PYG@tok@k\endcsname{\let\PYG@bf=\textbf\def\PYG@tc##1{\textcolor[rgb]{0.00,0.44,0.13}{##1}}}
\expandafter\def\csname PYG@tok@se\endcsname{\let\PYG@bf=\textbf\def\PYG@tc##1{\textcolor[rgb]{0.25,0.44,0.63}{##1}}}
\expandafter\def\csname PYG@tok@sd\endcsname{\let\PYG@it=\textit\def\PYG@tc##1{\textcolor[rgb]{0.25,0.44,0.63}{##1}}}

\def\PYGZbs{\char`\\}
\def\PYGZus{\char`\_}
\def\PYGZob{\char`\{}
\def\PYGZcb{\char`\}}
\def\PYGZca{\char`\^}
\def\PYGZam{\char`\&}
\def\PYGZlt{\char`\<}
\def\PYGZgt{\char`\>}
\def\PYGZsh{\char`\#}
\def\PYGZpc{\char`\%}
\def\PYGZdl{\char`\$}
\def\PYGZhy{\char`\-}
\def\PYGZsq{\char`\'}
\def\PYGZdq{\char`\"}
\def\PYGZti{\char`\~}
% for compatibility with earlier versions
\def\PYGZat{@}
\def\PYGZlb{[}
\def\PYGZrb{]}
\makeatother

\renewcommand\PYGZsq{\textquotesingle}

\begin{document}

\maketitle
\tableofcontents
\phantomsection\label{index::doc}


Contents:


\chapter{The OraMon4PRTG Reference}
\label{api:the-oramon4prtg-reference}\label{api::doc}\label{api:welcome-to-oramon4prtg-s-documentation}

\section{What is OraMon4PRTG?}
\label{api:what-is-oramon4prtg}
Is a set of scripts in Python to implement Oracle instances monitoring for the
paessler’s PRTG system included as an advanced ssh script


\section{Software requirements}
\label{api:software-requirements}\begin{itemize}
\item {} 
Python 2.7.x

\item {} 
cx\_Oracle 5.2

\item {} 
Red Hat Enterprise Linux Server release 5.8 or above

\item {} 
Oracle 11gR2 or above

\end{itemize}


\section{About the code (Technical)}
\label{api:about-the-code-technical}

\subsection{The config File}
\label{api:the-config-file}\label{api:module-config}\index{config (module)}
section for DB Connection info
\index{DbData (in module config)}

\begin{fulllineitems}
\phantomsection\label{api:config.DbData}\pysigline{\code{config.}\bfcode{DbData}\strong{ = \{`username': `SYSTEM', `database': `wms', `password': `orat3\$', `address': `127.0.0.1'\}}}
format for table spaces is {[}ChannelName,MaxWarning,MaxError{]}

\end{fulllineitems}

\index{config (class in config)}

\begin{fulllineitems}
\phantomsection\label{api:config.config}\pysigline{\strong{class }\code{config.}\bfcode{config}}
\end{fulllineitems}


The config file was designer so certain data into it can
startup the configuration for making OraMon4PRTG work
\begin{quote}
\index{DbData (in module config)}

\begin{fulllineitems}
\pysigline{\code{config.}\bfcode{DbData}}
\end{fulllineitems}

\index{TableSpaces (in module config)}

\begin{fulllineitems}
\phantomsection\label{api:config.TableSpaces}\pysigline{\code{config.}\bfcode{TableSpaces}}
\end{fulllineitems}

\end{quote}


\subsection{The OrMgr Library}
\label{api:module-OrMgr}\label{api:the-ormgr-library}\index{OrMgr (module)}\index{OrMgr (class in OrMgr)}

\begin{fulllineitems}
\phantomsection\label{api:OrMgr.OrMgr}\pysigline{\strong{class }\code{OrMgr.}\bfcode{OrMgr}}
Handling the Oracle connections
\index{db\_connect() (OrMgr.OrMgr method)}

\begin{fulllineitems}
\phantomsection\label{api:OrMgr.OrMgr.db_connect}\pysiglinewithargsret{\bfcode{db\_connect}}{}{}
Conect to the Oracle DB with Info from config.py

\end{fulllineitems}

\index{db\_close() (OrMgr.OrMgr method)}

\begin{fulllineitems}
\phantomsection\label{api:OrMgr.OrMgr.db_close}\pysiglinewithargsret{\bfcode{db\_close}}{}{}
disconect cursor and database

\end{fulllineitems}


\end{fulllineitems}



\subsection{The Checks Library}
\label{api:the-checks-library}\label{api:module-Checks}\index{Checks (module)}\index{Checks (class in Checks)}

\begin{fulllineitems}
\phantomsection\label{api:Checks.Checks}\pysigline{\strong{class }\code{Checks.}\bfcode{Checks}}
Multiple Predefined Monitoring SQL Querys
\index{ChkSize() (Checks.Checks method)}

\begin{fulllineitems}
\phantomsection\label{api:Checks.Checks.ChkSize}\pysiglinewithargsret{\bfcode{ChkSize}}{\emph{TableName}}{}
Check Size in MB of a Table

\end{fulllineitems}

\index{ChkRows() (Checks.Checks method)}

\begin{fulllineitems}
\phantomsection\label{api:Checks.Checks.ChkRows}\pysiglinewithargsret{\bfcode{ChkRows}}{\emph{TableName}}{}
no funciona query verificar con que usuario se hace

\end{fulllineitems}

\index{asm\_volume\_use() (Checks.Checks method)}

\begin{fulllineitems}
\phantomsection\label{api:Checks.Checks.asm_volume_use}\pysiglinewithargsret{\bfcode{asm\_volume\_use}}{\emph{name}}{}
Display Percentage usage of an ASM Table

\end{fulllineitems}

\index{ChkTblSpace() (Checks.Checks method)}

\begin{fulllineitems}
\phantomsection\label{api:Checks.Checks.ChkTblSpace}\pysiglinewithargsret{\bfcode{ChkTblSpace}}{\emph{TableName}}{}
Return Table space usage info, output Example:
TABLESPACE                USED\_MB    FREE\_MB   TOTAL\_MB   PCT\_FREE
---------------------- ---------- ---------- ---------- ----------
DCS\_D\_01                      263         31        294         11

\end{fulllineitems}


\end{fulllineitems}



\subsection{The XMLTags Library}
\label{api:module-XMLTags}\label{api:the-xmltags-library}\index{XMLTags (module)}\index{XMLTags (class in XMLTags)}

\begin{fulllineitems}
\phantomsection\label{api:XMLTags.XMLTags}\pysigline{\strong{class }\code{XMLTags.}\bfcode{XMLTags}}
This class handles the XMLtaging for PRTG's Advance Script
requirements
\index{getData() (XMLTags.XMLTags method)}

\begin{fulllineitems}
\phantomsection\label{api:XMLTags.XMLTags.getData}\pysiglinewithargsret{\bfcode{getData}}{\emph{Opt}, \emph{Val}}{}
this function reads an Option code (see self.Codes) and a
Value so it can append to the dictionary  tags and  return
as an XML code

\end{fulllineitems}

\index{S3p() (XMLTags.XMLTags method)}

\begin{fulllineitems}
\phantomsection\label{api:XMLTags.XMLTags.S3p}\pysiglinewithargsret{\bfcode{S3p}}{\emph{Cha}, \emph{Unt}, \emph{Val}}{}
simple 3 parameters (channel, unit, value)

\end{fulllineitems}

\index{ScU() (XMLTags.XMLTags method)}

\begin{fulllineitems}
\phantomsection\label{api:XMLTags.XMLTags.ScU}\pysiglinewithargsret{\bfcode{ScU}}{\emph{Cha}, \emph{CsU}, \emph{Val}}{}
simple Custom Unit

\end{fulllineitems}

\index{SFP() (XMLTags.XMLTags method)}

\begin{fulllineitems}
\phantomsection\label{api:XMLTags.XMLTags.SFP}\pysiglinewithargsret{\bfcode{SFP}}{\emph{Cha}, \emph{Val}}{}
Simple Float Percent

\end{fulllineitems}

\index{TblSpcs() (XMLTags.XMLTags method)}

\begin{fulllineitems}
\phantomsection\label{api:XMLTags.XMLTags.TblSpcs}\pysiglinewithargsret{\bfcode{TblSpcs}}{\emph{TS}, \emph{Val}}{}
TS = {[}ChanelName,MaxWarning,Maxerror{]},Value

\end{fulllineitems}


\end{fulllineitems}



\section{The Main File}
\label{api:the-main-file}
\emph{OraMon4PRTG requiers a main program which will be executed by the
advance ssh script, so now we will show two examples of the usage
and development of this final script.}


\subsection{Example: The functions caller}
\label{api:example-the-functions-caller}

\subsection{Example: The table Spaces Example}
\label{api:example-the-table-spaces-example}

\chapter{OraMon4PRTG Fast Install}
\label{FastInstall:oramon4prtg-fast-install}\label{FastInstall::doc}
NOTE: \textbf{For doing this you must verify you're logged in as root user}


\section{Create folder path into the system}
\label{FastInstall:create-folder-path-into-the-system}
\begin{Verbatim}[commandchars=\\\{\}]
\PYG{g+gp}{\PYGZgt{}\PYGZgt{}\PYGZgt{} }\PYG{n}{mkdir} \PYG{o}{/}\PYG{n}{var}\PYG{o}{/}\PYG{n}{prtg}
\PYG{g+gp}{\PYGZgt{}\PYGZgt{}\PYGZgt{} }\PYG{n}{mkdir} \PYG{o}{/}\PYG{n}{var}\PYG{o}{/}\PYG{n}{prtg}\PYG{o}{/}\PYG{n}{scriptsxml}
\end{Verbatim}


\section{Clone with git the program files}
\label{FastInstall:clone-with-git-the-program-files}
\begin{Verbatim}[commandchars=\\\{\}]
\PYG{g+gp}{\PYGZgt{}\PYGZgt{}\PYGZgt{} }\PYG{n}{cd} \PYG{o}{/}\PYG{n}{var}\PYG{o}{/}\PYG{n}{prtg}\PYG{o}{/}\PYG{n}{scriptsxml}
\PYG{g+gp}{\PYGZgt{}\PYGZgt{}\PYGZgt{} }\PYG{n}{git} \PYG{n}{clone} \PYG{n}{ruta}
\end{Verbatim}


\section{Give PRTG user permissions}
\label{FastInstall:give-prtg-user-permissions}
\begin{Verbatim}[commandchars=\\\{\}]
\PYG{g+gp}{\PYGZgt{}\PYGZgt{}\PYGZgt{} }\PYG{n}{chown} \PYG{o}{\PYGZhy{}}\PYG{n}{Rv} \PYG{n}{user}\PYG{o}{.}\PYG{n}{group} \PYG{o}{/}\PYG{n}{var}\PYG{o}{/}\PYG{n}{prtg}
\end{Verbatim}

where the \textbf{user} and \textbf{group} are for the monitor user you want to use.

\begin{Verbatim}[commandchars=\\\{\}]
\PYG{g+gp}{\PYGZgt{}\PYGZgt{}\PYGZgt{} }\PYG{n}{chmod} \PYG{n}{u}\PYG{o}{+}\PYG{n}{x} \PYG{o}{/}\PYG{n}{var}\PYG{o}{/}\PYG{n}{prtg}\PYG{o}{/}\PYG{n}{scriptsxml}\PYG{o}{/}\PYG{n}{OraMon4PRTG}\PYG{o}{/}\PYG{n}{Main}\PYG{o}{.}\PYG{n}{py}
\end{Verbatim}


\section{Edit config file}
\label{FastInstall:edit-config-file}
Edit the config.py file so that you can add at dictionary Database
\begin{itemize}
\item {} 
username for Oracle.

\item {} 
password for Oracle.

\item {} 
IP address where the Oracle DB is hosted.

\item {} 
Oracle database name

\end{itemize}

for further information see The
{\hyperref[api:module-config]{\emph{\code{config}}}}
file


\section{Add ssh script advance into prtg}
\label{FastInstall:add-ssh-script-advance-into-prtg}

\chapter{Indices and tables}
\label{index:indices-and-tables}\begin{itemize}
\item {} 
\DUspan{xref,std,std-ref}{genindex}

\item {} 
\DUspan{xref,std,std-ref}{modindex}

\item {} 
\DUspan{xref,std,std-ref}{search}

\end{itemize}


\renewcommand{\indexname}{Python Module Index}
\begin{theindex}
\def\bigletter#1{{\Large\sffamily#1}\nopagebreak\vspace{1mm}}
\bigletter{c}
\item {\texttt{Checks}}, \pageref{api:module-Checks}
\item {\texttt{config}}, \pageref{api:module-config}
\indexspace
\bigletter{o}
\item {\texttt{OrMgr}}, \pageref{api:module-OrMgr}
\indexspace
\bigletter{x}
\item {\texttt{XMLTags}}, \pageref{api:module-XMLTags}
\end{theindex}

\renewcommand{\indexname}{Index}
\printindex
\end{document}
